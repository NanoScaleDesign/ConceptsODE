\documentclass[12pt,letterpaper,twoside]{amsart}
\usepackage[latin1]{inputenc}
\usepackage{amsmath}
\usepackage{amsfonts}
\usepackage{graphicx}
\usepackage{amssymb}
\usepackage{multicol}
\usepackage{ulem}
\newcounter{example}
\newcounter{exercise}
\newcounter{problem}
\newtheorem{theorem}{Theorem}
\newcommand{\example}{\bigskip \noindent {\large {\sc Example \arabic{example}:}} \addtocounter{example}{1}}
\newcommand{\exercise}{\bigskip \noindent {\large {\sc Exercise \arabic{exercise}:}} \addtocounter{exercise}{1}}
\newcommand{\problem}{\bigskip \noindent {\large {\sc Problem \arabic{problem}:}} \addtocounter{problem}{1}}
\newcommand{\tech}{\marginpar{\vskip 10mm \begin{center}\includegraphics[width=0.25in]{calculatorimagesmall.eps} \end{center}}}
\newcommand{\solution}{\medskip \noindent {\bf Solution: }}
\newcommand{\R}{\mathbb{R}}





\begin{document}

\sffamily

%%%%  switch the commenting on this line and the next \chapter{Introduction}
\begin{center} {\LARGE Complex Numbers} \end{center}

\setcounter{example}{1}
\setcounter{exercise}{1}

When we solve characteristic equations, we are often faced with complex numbers.  For example, the solutions of a quadratic equation $ax^2+bx+c=0$ are given by the quadratic formula
\[ x = \frac{-b \pm \sqrt{b^2-4ac}}{2a},\]
and if the discriminant $b^2-4ac$ is negative, then we are looking at square roots of negative numbers, so the roots of the original equation are complex numbers -- these are numbers which can be written in the form
\[ z = \alpha + \beta i,\]
where $\alpha$ and $\beta$ are both real, and $i$ satisfies $i^2=-1$.  If $z=\alpha + \beta i$ in this way, then we call $\alpha$ the {\bf real part} of $z$, and we call $\beta$ the {\bf imaginary part}.

We can usually understand the arithmetic operations on complex numbers by writing numbers in terms of their real and imaginary parts.

\example Let $z=1+3i$ and $w=3-2i$.  Then:
\begin{itemize}
\item $z+w = (1+3i)+(3-2i)=(1+3)+(3-2)i=4+i$
\item $zw = (1+3i)(3-2i)=(1)(3)+(3i)(3)+(1)(-2i)+(3i)(-2i)=3-2i+3i-6i^2=3-2i+3i-6(-1)=9+i$
\end{itemize}

\exercise Let $u=2+4i$ and $v=1-2i$.  Find $2u+3v$ and $2uv$.

The {\bf complex conjugate} of $z=\alpha + \beta i$ is the complex number $\overline{z}=\alpha - \beta i$.  For example, $\overline{1+3i}=1-3i$.  

\exercise prove that for any complex number $z$, the product $z \cdot \overline{z}$ is a real number.

\medskip
The conjugate is thus useful for simplifying division with complex numbers:

\example Let $z=1+2i$ and $w=2-4i$.  Then
\begin{align*}
\frac{z}{w} & = \frac{1+2i}{2-4i} \\
& = \frac{(1+2i)}{(2-4i)} \frac{(2+4i)}{(2+4i)} \\
& = \frac{2+4i+4i+8i^2}{4-16i^2} \\
& = \frac{-6+8i}{20} \\
& = -\frac{3}{10}+\frac{2}{5}i.
\end{align*}

\exercise Let $u=2+4i$ and $v=1-i$.  Simplify the expressions $\frac{u}{v}$ and $\frac{v}{u}$.  Write the answers in the form $\alpha + \beta i$.

\medskip
In the study of ordinary differential equations, we will often see complex numbers arise in exponential functions. Therefore we now turn our attention to finding a better understanding of exponentials.

First of all, we need to say what we mean by $e^z$ when $z$ is complex.  To answer this, we turn to the power series representation of the exponential function:
\[ e^z = \sum_{n=0}^\infty \frac{z^n}{n!}.\]
(Here, we use the standard convention when working with power series that $0^0=1$.)  This series has an infinite radius of convergence and therefore converges for all complex numbers $z$.

To work with a complex exponent, we usually write it in terms of its real and imaginary parts, and then use a law of exponents to separate these:
\[ e^z = e^{\alpha + \beta i} = e^\alpha e^{\beta i}.\]
Therefore it will be profitable for us if we now focus our attention on expressions of the form $e^{\beta i}$, and that's where the power series representation becomes helpful:
\begin{align*}
e^{\beta i} & = \sum_{n=0}^\infty \frac{(\beta i)^n}{n!} \\
& = \sum_{n=0}^\infty \frac{\beta^n i^n}{n!} \\
& = \sum_{n=0, \ n \ even}^\infty \frac{\beta^n i^n}{n!} + \sum_{n=0, \ n \ odd}^\infty \frac{\beta^n i^n}{n!} \\
& = \sum_{n=0}^\infty \frac{\beta^{2n}i^{2n}}{(2n)!}+\sum_{n=0}^\infty \frac{\beta^{2n+1}i^{2n+1}}{(2n+1)!} \\
& = \sum_{n=0}^\infty \frac{\beta^{2n}(-1)^n}{(2n)!}+\sum_{n=0}^\infty \frac{\beta^{2n+1}i(-1)^n}{(2n+1)!} \\
& = \cos(\beta)+i\sin(\beta).
\end{align*}
Combining this with the previous result gives us:
\begin{center}
\fbox{
\begin{minipage}{2in}
\[ e^{\alpha + \beta i} = e^\alpha (\cos(\beta)+i\sin(\beta))\]
\end{minipage}
}
\end{center}

Note that in the calculation above, we made use of the power series for sine and cosine:
\[ \sin(z)=\sum_{n=0}^\infty \frac{z^{2n+1}(-1)^n}{(2n+1)!} \ \ \ \mbox{and} \ \ \ \cos(z) = \sum_{n=0}^\infty \frac{z^{2n} (-1)^n}{(2n)!}\]
These series also allow us to define sine and cosine for complex arguments, and this will be explored briefly in the problem set.

\exercise Find the values of $e^{\pi i}$, $e^{2\pi i}$ and $e^{2+\pi i/4}$.


\bigskip

%% Cut below here for the book form.

\begin{center} {\LARGE Problems} \end{center}

\setcounter{problem}{1}

\problem Prove that, if the solutions of the quadratic equation $ax^2+bx+c=0$ are complex numbers, and if the coefficients $a$, $b$ and $c$ are all real numbers, then the solutions are complex conjugates of one another.

\problem Prove that $\sin(z)=\frac{e^{-iz}-e^{iz}}{2}$.  Use this to evaluate $\sin(i)$.

\problem Find a representation formula for $\cos(z)$ (similar to the one for sine above) and use it to evaluate $\cos(2i)$.





\end{document}